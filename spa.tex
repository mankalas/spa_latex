\documentclass{article}

\usepackage{amsmath}
\usepackage{siunitx}
\usepackage{mathtools}
\usepackage{url}
\usepackage{makeidx} \makeindex

\DeclarePairedDelimiter\ceil{\lceil}{\rceil}
\DeclarePairedDelimiter\floor{\lfloor}{\rfloor}
\DeclareMathOperator{\arctantwo}{arctan2}

\begin{document}

\tableofcontents

\section{Introduction}

\par With the continuous technological advancements in solar radiation
applications, there will always be a demand for smaller uncertainty in
calculating the solar position. Many methods to calculate the solar
position have been published in the solar radiation literature,
nevertheless, their uncertainties have been greater than
$\pm$~\ang{0.01} in solar zenith and azimuth angle calculations, and
some are only valid for a specific number of years \cite{blanco}. \par
For example, Michalsky's calculations are limited to the period from
1950 to 2050 with uncertainty of greater than $\pm$~\ang{0.01}
\cite{michalsky}, and the calculations of Blanco-Muriel et al.'s are
limited to the period from 1999 to 2015 with uncertainty greater than
$\pm$~\ang{0.01} \cite{blanco}.\\ %

An example emphasizing the importance of reducing the uncertainty of
calculating the solar position to lower than $\pm$~\ang{0.01}, is the
calibration of pyranometers that measure the global solar
irradiance. During the calibration, the responsivity of the
pyranometer is calculated at zenith angles from \ang{0} to \ang{90} by
dividing its output voltage by th e reference global solar irradiance
($G$), which is a function of the cosine of the zenith angle
($\cos(\Theta)$). Figure 1 shows the magnitude of errors that the
\ang{0.01} uncertainty in $\Theta$ can contribute to the calculation of
$\cos(\Theta)$ , and consequently $G$ that is used to calculate the
responsivity.  Figure 1 shows that the uncertainty in $\cos(\Theta)$
exponentially increases as 2 reaches \ang{90} (e.g. at $\Theta$ equal to
\ang{87}, the uncertainty in $\cos(\Theta)$ is $0.7\%$, which can result in
an uncertainty of 0.35\% in calculating $G$; because at such large
zenith angles the normal incidence irradiance is approximately equal
to half the value of $G$). From this arises the need to use a solar
position algorithm with lower uncertainty for users that are
interested in measuring the global solar irradiance with smaller
uncertainties in the full zenith angle range from \ang{0} to
\ang{90}.\\ %


\par In this report we describe a procedure for a Solar Position
Algorithm (SPA) to calculate the solar zenith and azimuth angle with
uncertainties equal to $\pm$~\ang{0.0003} in the period from the year
-2000 to 6000. The procedure is adopted from \textit{The Astronomical
  Algorithms} \cite{meeus}, which is based on the Variations
Séculaires des Orbites Planétaires Theory (VSOP87) that was developed
by P. Bretagnon in 1982 then modified in 1987 by Bretagnon and Francou
\cite{meeus}. In this report, we summarize the complex algorithm
elements scattered throughout the book to calculate the solar
position, and introduce some modification to the algorithm to
accommodate solar radiation applications. For example, in \textit{The
  Astronomical Algorithms} \cite{meeus}, the azimuth angle is measured
westward from south, but for solar radiation applications, it is
measured eastward from north. Also, the observer’s geographical
longitude is considered positive west, or negative east from
Greenwich, while for solar radiation applications, it is considered
negative west, or positive east from Greenwich.\\ %

\par We start this report by:
\begin{itemize}
\item Describing the time scales because of the importance of using
  the correct time in the SPA;
\item Providing a step by step procedure to calculate the solar
  position and the solar incidence angle for an arbitrary surface
  orientation using the methods described in \textit{An Introduction
    to Solar Radiation} \cite{iqbal};
\item Evaluating the SPA against the \textit{Astronomical Almanac}
  (AA) data for the years 1994, 1995, 1996, and 2004.
\end{itemize}

\par Because of the complexity of the algorithm we included some
examples, in the Appendix, to give the users confidence in their step
by step calculations. We also included in the Appendix an explanation
of how to calculate the equation of time, sun transit (solar noon),
sunrise, sunset, and how to change the Julian Day to a Calendar
Date. We also included a C source code with header file, for all the
calculations in this report (except for the Julian Day to Calendar
Date conversion). The users can incorporate this module into their own
code by including the header file, declaring the SPA structure,
filling in the required input parameters into the strucWeture, and
then call the SPA calculation function. This function will calculate
all the output values and fill in the SPA structure for the user.\\ %

\par The users should note that this report is used to calculate the
solar position for solar radiation applications only, and that it is
purely mathematical and not meant to teach astronomy or to describe
the Earth rotation. For more description about the astronomical
nomenclature that is used through out the report, the user is
encouraged to review the definitions in the \textit{Astronomical
  Almanacs}, or other astronomical reference.

\section{Time Scale}

\par The following are the internationally recognized time scales:

\begin{itemize}
\item The Universal Time ($UT$), or Greenwich civil time, is based on
  the Earth’s rotation and counted from 0-hour at midnight; the unit
  is mean solar day \cite{meeus}. $UT$ is the time used to calculate
  the solar position in the described algorithm. It is sometimes
  referred to as UT1.
\item The International Atomic Time ($TAI$) is the duration of the
  System International Second (SI-second) and based on a large number
  of atomic clocks \cite{norwich}.
\item The Coordinated Universal Time ($UTC$) is the bases of most
  radio time signals and the legal time systems. It is kept to within
  0.9 seconds of $UT1$ ($UT$) by introducing one second steps to its
  value (leap second); to date the steps are always positive.
\item The Terrestrial Dynamical or Terrestrial Time ($TDT$ or $TT$) is
  the time scale of ephemerides for observations from the Earth
  surface.
\end{itemize}

\par The following equations describe the relationship between the above
time scales (in seconds):

\begin{equation}
  \label{eq:tt}
  TT = TAI + 32.184
\end{equation}

\begin{equation}
  \label{eq:ut}
  UT = TT - \Delta T
\end{equation}

\par where $\Delta T$ is the difference between the Earth rotation time and
the Terrestrial Time ($TT$). It is derived from observation only and
reported yearly in the Astronomical Almanac [5].

\begin{equation}
  \label{eq:ut1}
  UT = UT1 = UTC + \Delta UT1
\end{equation}

\par where $\Delta UT1$ is a fraction of a second, positive or negative value,
that is added to the $UTC$ to adjust for the Earth irregular
rotational rate. It is derived from observation, but predicted values
are transmitted in code in some time signals, e.g. weekly by the
U.S. Naval Observatory (USNO) [6].

\section{Procedure}

\subsection{Calculate the Julian and Julian Ephemeris Day, Century,
  and Millennium}

\par The Julian date starts on January 1, in the year - 4712 at 12:00:00
UT. The Julian Day ($JD$) is calculated using UT and the Julian
Ephemeris Day ($JDE$) is calculated using $TT$. In the following
steps, note that there is a 10-day gap between the Julian and
Gregorian calendar where the Julian calendar ends on October 4, 1582
($JD = 2299160$), and after 10-days the Gregorian calendar starts on
October 15, 1582.

\subsubsection{Calculate the Julian Day ($JD$)}

\begin{equation}
  \label{eq:jd}
  JD = \floor{365.25 \times (Y + 4716)} + \floor{30.6001 \times (M + 1)} + D + B - 1524.5
\end{equation}

\par where
\begin{itemize}
\item $\floor{x}$ is the integer of the calculated term (e.g. 8.7~=~8,
  8.2~=~8, and -8.7~` ­ 8, etc.);
\item $Y$ is the year (e.g. 2001, 2002, 2018, etc.);
\item $M$ is the month of the year (e.g. 1 for January, etc.). Note
  that if $M = 1$ or $M = 2$, then $Y = Y - 1$ and $M = M + 12$. If
  $M > 2$, then $Y$ and $M$ are not changed;
\item $D$ is the day of the month with decimal time (e.g. for the
  second day of the month at 12:30:30 UT, $D = 2.521180556$);
\item $B$ is equal to
  \begin{itemize}
  \item $0$, for the Julian calendar (i.e. by using $B = 0$ in equation
    \ref{eq:jd}, $JD < 2299160$);
  \item $(2 - A + \floor{A/4})$) for the Gregorian calendar (i.e. by
    using $B = 0$ in equation \ref{eq:jd}, $JD > 2299160$), where $A
    = \floor{Y/100}$.
  \end{itemize}
\end{itemize}

\par For users who wish to use their local time instead of UT, change the
time zone to a fraction of a day (by dividing it by 24), then subtract
the result from $JD$. Note that the fraction is subtracted from $JD$
calculated before the test for $B < 2299160$ to maintain the Julian
and Gregorian periods.

\par Table \ref{tbl:jd} shows examples to test any implemented program
used to calculate the $JD$.

\subsubsection{Calculate the Julian Ephemeris Day ($JDE$)}

\begin{equation}
  \label{eq:jde}
  JDE = JD + \frac{\Delta T}{86400}
\end{equation}

\subsubsection{Calculate the Julian century ($JC$) and the Julian
  Ephemeris Century ($JCE$) for the 2000 standard epoch}

\begin{equation}
  \label{eq:jc}
  JC = \frac{JD - 2451545}{36525}
\end{equation}

\begin{equation}
  \label{eq:jce}
  JCE = \frac{JDE - 2451545}{36525}
\end{equation}

\subsubsection{Calculate the Julian Ephemeris Millennium ($JME$) for
  the 2000 standard epoch}

\begin{equation}
  \label{eq:jme}
  JME = \frac{JCE}{10}
\end{equation}

\subsection{Calculate the Earth heliocentric longitude, latitude, and
  radius vector ($L$, $B$, and $R$)}
\label{sec:earth_heliocentric}

\par “Heliocentric” means that the Earth position is calculated with
respect to the center of the sun.

\begin{enumerate}

\item \label{item:step_1} For each row of table
  \ref{tbl:earth_periodic_terms}, calculate the term $L0_i$ (in
  radians):

  \begin{equation}
    \label{eq:l0i}
    L0_i = A_i \times \cos\left(B_i + C_i \times JME\right)
  \end{equation}
  where
  \begin{itemize}
  \item $i$ is the $i$th row for the term $L0$ in table
    \ref{tbl:earth_periodic_terms};
  \item $A_i$, $B_i$ and $C_i$ are the values in the $i$th row and
    $A$, $B$ and $C$ columns in the table
    \ref{tbl:earth_periodic_terms}, for the term $L0$ (in radians).
  \end{itemize}

\item Calculate the term $L0$ (in radians):

  \begin{equation}
    \label{eq:l0}
    L0 = \sum_{i=0}^{n}L0_i
  \end{equation}
  \par where $n$ is the number of rows for the term $L0$ in table
  \ref{tbl:earth_periodic_terms}.

\item \label{item:step_3} Calculate the terms $L1$, $L2$, $L3$, $L4$,
  and $L5$ by using equations \ref{eq:l0i} and \ref{eq:l0} and
  changing the 0 to 1, 2, 3, 4, and 5, and by using their
  corresponding values in 4 columns $A$, $B$, and $C$ in table
  \ref{tbl:earth_periodic_terms} (in radians);

\item \label{item:step_4} Calculate the Earth heliocentric longitude,
  $L_r$ (in radians),

  \begin{equation}
    \label{eq:l}
    L_r = \frac{L0 + L1 \times JME + L2 \times JME^2 + L3 \times JME^3 + L4 \times JME^4 + L5 \times JME^5}{10^8}
  \end{equation}

\item \label{item:step_5} Calculate $L$ in degrees

  \begin{equation}
    \label{eq:l_in_degrees}
    L = \frac{L_r \times 180}{\pi}
  \end{equation}

  \par where $\pi$ is approximately equal to 3.1415926535898;

\item \label{tiem:step_6} Limit $L$ to the range from \ang{0} to
  \ang{360}. That can be accomplished by dividing $L$ by 360 and
  recording the decimal fraction of the division as $F$. If $L$ is
  positive, then the limited $L = 360 * F$ .If $L$ is negative, then
  the limited $L = 360 - 360 * F$;

\item \label{item:step_7} Calculate the Earth heliocentric latitude,
  $B$ (in degrees), by using table \ref{tbl:earth_periodic_terms} and
  steps \ref{item:step_1} through \ref{item:step_5} and by replacing
  all the $L$s by $B$s in all equations. Note that there are no $B2$
  through $B5$, consequently, replace them by zero in steps
  \ref{item:step_3} and \ref{item:step_4};

\item Calculate the Earth radius vector, $R$ (in Astronomical Units,
  AU), by repeating step \ref{item:step_7} and by replacing all $L$s
  by $R$s in all equations. Note that there is no $R5$, consequently,
  replace it by zero in steps \ref{item:step_3} and \ref{item:step_4}.

\end{enumerate}

\subsection{Calculate the geocentric longitude and latitude ($\Theta$ and
  $\beta$)}

\par “Geocentric” means that the sun position is calculated with respect
to the Earth center.

\begin{enumerate}
\item Calculate the geocentric longitude, $\Theta$ (in degrees)

  \begin{equation}
    \label{eq:theta}
    \Theta = L + 180
  \end{equation}

\item Limit $\Theta$ to the range from \ang{0} to \ang{360} as described in
  step \ref{item:step_6} in section \ref{sec:earth_heliocentric};

\item Calculate the geocentric latitude, $\beta$ (in degrees)

  \begin{equation}
    \label{eq:geo_lat}
    \beta = -B
  \end{equation}
\end{enumerate}

\subsection{Calculate the nutation in longitude and obliquity
  ($\Delta\Psi$ and $\Delta\varepsilon$)}

\begin{enumerate}
\item Calculate the mean elongation of the moon from the sun, $X0$ (in
  degrees)
  \begin{equation}
    \label{eq:x0}
    X0 = 297.85036 + 445267.111480 \times JCE - 0.0019142 \times JCE^2 + \frac{JCE^3}{189474}
  \end{equation}

\item Calculate the mean anomaly of the sun (Earth), $X1$ (in degrees)
  \begin{equation}
    \label{eq:x1}
    X1 = 357.52772 + 35999.050340 \times JCE - 0.0001603 \times JCE^2 - \frac{JCE^3}{300000}
  \end{equation}

\item Calculate the mean anomaly of the moon, $X2$ (in degrees),
  \begin{equation}
    \label{eq:x2}
    X2 = 134.96298 + 477198.867398 \times JCE + 0.0086972 \times JCE^2 + \frac{JCE^3}{56250}
  \end{equation}

\item Calculate the moon’s argument of latitude, $X3$ (in degrees),
  \begin{equation}
    \label{eq:x3}
    X3 = 93.27191 + 483202.017538 \times JCE - 0.0036825 \times JCE^2 + \frac{JCE^3}{327270}
  \end{equation}

\item Calculate the longitude of the ascending node of the moon’s mean
  orbit on the ecliptic, measured from the mean equinox of the date,
  $X4$ (in degrees),
  \begin{equation}
    \label{eq:x4}
    X4 = 125.04452 - 1934.136261 \times JCE + 0.0020708 \times JCE^2 + \frac{JCE^3}{450000}
  \end{equation}

\item For each row of table \ref{tbl:nutation_periodic_terms},
  calculate the terms $\Delta\Psi$ and $\Delta\varepsilon$ (in 0.0001of arc seconds)

  \begin{equation}
    \label{eq:delta_psi_i}
    \Delta\Psi_i = (a_i + b_i \times JCE) \times \sin\left(\sum_{j=0}^4 X_j \times Y_{i,j}\right)
  \end{equation}

  \begin{equation}
    \label{eq:delta_epsilon_i}
    \Delta\varepsilon_i = (c_i + d_i \times JCE) \times \cos\left(\sum_{j=o}^4 X_j \times Y_{i,j}\right)
  \end{equation}

  \par where
  \begin{itemize}
  \item $a_i$, $b_i$, $c_i$ and $d_i$ are the values listed in the
    $i$th row and columns $a$, $b$, $c$ and $d$ in table
    \ref{tbl:nutation_periodic_terms};
  \item $X_i$ is the $j$th $X$ calculated by using equation
    \ref{eq:x0} through \ref{eq:x4};
  \item $Y_{i,j}$ is the value listed in the $i$th row and $j$th $Y$
    column in table \ref{tbl:nutation_periodic_terms}.
  \end{itemize}

\item Calculate the nutation longitude $\Delta\Psi$ (in degrees)
  \begin{equation}
    \label{eq:delta_psi}
    \Delta\Psi = \frac{\sum_{i=0}^n\Delta\Psi_i}{36000000}
  \end{equation}

  \par where $n$ in the number of rows in table
  \ref{tbl:nutation_periodic_terms} ($n = 63$ rows in the table)

\item Calculate the nutation in obliquity $\Delta\varepsilon$ (in degrees)
  \begin{equation}
    \label{eq:delta_epsilon}
    \Delta\varepsilon = \frac{\sum_{i=0}^n\Delta\varepsilon_i}{36000000}
  \end{equation}
\end{enumerate}

\subsection{Calculate the true obliquity of the ecliptic $\varepsilon$ (in
  degrees)}

\begin{enumerate}
\item Calculate the mean obliquity of the ecliptic $\varepsilon_0$ (in arc
  seconds)
  \begin{equation}
    \label{eq:epsilon_0}
    \begin{split}
      \varepsilon_0 = & 84381.448 - 4680.93U - 1.55U^2 + 1999.25U^3 - \\
      & 51.38U^4 - 249.67U^5 - 39.05U^6 + 7.12U^7 + \\
      & 27.87U^8 + 5.79U^9 + 2.45U^{10}
    \end{split}
  \end{equation}

  \par where $U = \frac{JME}{10}$;

\item Calculate the true obliquity of the ecliptic, $\varepsilon$ (in degrees),

  \begin{equation}
    \label{eq:epsilon_0}
    \varepsilon = \frac{\varepsilon_0}{3600} + \Delta\varepsilon
  \end{equation}
\end{enumerate}

\subsection{Calculate the aberration correction, $\Delta\tau$ (in degrees)}

\begin{equation}
  \label{eq:delta_t}
  \Delta\tau = -\frac{20.4898}{3600 \times R}
\end{equation}

\subsection{Calculate the apparent sun longitude, $\lambda$ (in degrees)}

\begin{equation}
  \label{eq:lambda}
  \lambda = \Theta + \Delta\Psi + \Delta\tau
\end{equation}

\subsection{Calculate the apparent sidereal time at Greenwich at any
  given time, $\nu$ (in degrees)}

\begin{enumerate}
\item Calculate the mean sidereal time at Greenwich, $\nu_o$ (in
  degrees)

  \begin{equation}
    \label{eq:nu_0}
    \nu_0 = 80.46061837 + 360.98564736629 \times (JD - 2451545) + 0.000387933 \times JC^2 - \frac{JC^3}{38710000}
  \end{equation}

\item Limit $\nu_0$ to the range from \ang{0} to \ang{360} as described
  in step \ref{item:step_6} in section \ref{sec:earth_heliocentric};

\item Calculate the apparent sidereal time at Greenwich, $\nu$ (in
  degrees,

  \begin{equation}
    \label{eq:nu}
    \nu = \nu_0 + \Delta\Psi \times \cos(\varepsilon)
  \end{equation}
\end{enumerate}

\subsection{Calculate the geocentric sun right ascension, $\alpha$ (in
  degrees}

\begin{enumerate}
\item Calculate the sun right ascension, $\alpha$ (in radians),
  \begin{equation}
    \label{eq:alpha}
    \alpha = \arctan2\left(\frac{\sin(\lambda) \times \cos(\varepsilon) - \tan(\beta) \times \sin(\varepsilon)}{\cos(\lambda)}\right)
  \end{equation}

  \par where $\arctan2$ is an arctangent function that is applied to the
  numerator and the denominator (instead of the actual division) to
  maintain the correct quadrant of the $\alpha$ where $\alpha$ is in the range
  from $-\pi$ to $\pi$.
\end{enumerate}

\begin{equation}
  \label{eq:delta}
  \delta = \arcsin\left(\sin(\beta) \times \cos(\varepsilon) + \cos(\beta) \times \sin(\varepsilon) \times \sin(\lambda)\right)
\end{equation}

\begin{equation}
  \label{eq:delta_alpha}
  \Delta\alpha = \arctan2\left(\frac{-x \times \sin(\xi) \times \sin(H)}{\cos(\delta) - x \times \sin(\xi) \times \cos(H)}\right)
\end{equation}

\begin{equation}
  \label{eq:delta_e}
  \Delta e = \frac{P}{1010}\times\frac{283}{273 + T} \times \frac{1.02}{60 \times \tan\left(e_0 + \frac{10.3}{e_0 + 5.11}\right)}
\end{equation}

\appendix

\printindex

\bibliographystyle{ieeetr}
\bibliography{spa}

\end{document}
